\chapter{RazOSin järjestelmäkutsut}

\label{Syscalls}

RazOSissa on kirjoitushetkellä 18 järjestelmäkutsua. Ne on määritelty tiedostossa \texttt{razos/razos\_kernel\_headers/api/razos.h}.

\textbf{sys\_exit}\\
Numero: 0\\
Kutsuminen: \texttt{void exit(int status)}\\
Implementaatio: \texttt{razos/kernel/src/syscall/sys\_tasking.c}\\
Kuvaus: lopettaa prosessin

\textbf{sys\_sched\_yield}\\
Numero: 1\\
Kutsuminen: \texttt{void sched\_yield(void)}\\
Implementaatio: \texttt{razos/kernel/src/syscall/sys\_tasking.c}\\
Kuvaus: luovuttaa suoritusvuoron

\textbf{sys\_fork}\\
Numero: 2\\
Kutsuminen: \texttt{pid\_t fork(void)}\\
Implementaatio: \texttt{razos/kernel/src/syscall/sys\_tasking.c}\\
Kuvaus: jakaa prosessin kahteen samanlaiseen

\textbf{sys\_wait}\\
Numero: 3\\
Kutsuminen: \texttt{uint32\_t wait(int* stat\_loc)}\\
Implementaatio: \texttt{razos/kernel/src/syscall/sys\_tasking.c}\\
Kuvaus: asettaa vanhemman odottamaan lapsiprosessin valmistumista

\textbf{sys\_read}\\
Numero: 4\\
Kutsuminen: \texttt{ssize\_t read(int fd, void* buf, size\_t size)}\\
Implementaatio: \texttt{razos/kernel/src/syscall/sys\_fs.c}\\
Kuvaus: lukee tiedostosta

\textbf{sys\_write}\\
Numero: 5\\
Kutsuminen: \texttt{ssize\_t write(int fd, const void* buf, size\_t size)}\\
Implementaatio: \texttt{razos/kernel/src/syscall/sys\_fs.c}\\
Kuvaus: kirjoittaa tiedostoon

\textbf{sys\_open}\\
Numero: 6\\
Kutsuminen: \texttt{int open(const char* name, int oflag, ...)}\\
Implementaatio: \texttt{razos/kernel/src/syscall/sys\_fs.c}\\
Kuvaus: avaa tiedoston

\textbf{sys\_close}\\
Numero: 7\\
Kutsuminen: \texttt{int close(int fd)}\\
Implementaatio: \texttt{razos/kernel/src/syscall/sys\_fs.c}\\
Kuvaus: sulkee avoimen tiedoston

\textbf{sys\_creat}\\
Numero: 8\\
Kutsuminen: \texttt{int creat(const char* name, mode\_t mode)}\\
Implementaatio: \texttt{razos/kernel/src/syscall/sys\_fs.c}\\
Kuvaus: luo uuden tiedoston

\textbf{sys\_lseek}\\
Numero: 9\\
Kutsuminen: \texttt{off\_t lseek(int fd, off\_t offset, int whence)}\\
Implementaatio: \texttt{razos/kernel/src/syscall/sys\_fs.c}\\
Kuvaus: siirtyy tiedostossa \textit{offset} tavua

\textbf{sys\_fcntl}\\
Numero: 10\\
Kutsuminen: \texttt{int fcntl(int fd, int cmd, ...)}\\
Implementaatio: \texttt{razos/kernel/src/syscall/sys\_fs.c}\\
Kuvaus: hakee tai asettaa monia tiedoston asetuksia/tietoja

\textbf{sys\_fstat}\\
Numero: 11\\
Kutsuminen: \texttt{int fstat(int fd, struct stat* buf)}\\
Implementaatio: \texttt{razos/kernel/src/syscall/sys\_fs.c}\\
Kuvaus: hakee tiedoston status-tietoja

\textbf{sys\_setup}\\
Numero: 12\\
Kutsuminen: \texttt{}\\
Implementaatio: \texttt{razos/kernel/src/syscall/sys\_setup.c}\\
Kuvaus: asettaa tai hakee kernelin asetuksia

\textbf{sys\_pipe}\\
Numero: 13\\
Kutsuminen: \texttt{int pipe(int fd[2])}\\
Implementaatio: \texttt{razos/kernel/src/syscall/sys\_pipe.c}\\
Kuvaus: luo pipen

\textbf{sys\_brk}\\
Numero: 14\\
Kutsuminen: \texttt{int brk(void* addr)}\\
Implementaatio: \texttt{razos/kernel/src/syscall/sys\_uvm.c}\\
Kuvaus: asettaa käyttäjän heap-alueen lopun osoitteeseen \textit{addr}

\textbf{sys\_sbrk}\\
Numero: 15\\
Kutsuminen: \texttt{void* sbrk(intptr\_t incr)}\\
Implementaatio: \texttt{razos/kernel/src/syscall/sys\_uvm.c}\\
Kuvaus: kasvattaa tai pienetää käyttäjän heap-aluetta \textit{incr} tavun verran

\textbf{sys\_execve}\\
Numero: 16\\
Kutsuminen: \texttt{int execve(const char* path, char* const argv[], char* const envp[])}\\
Implementaatio: \texttt{razos/kernel/src/syscall/sys\_execve.c}\\
Kuvaus: vaihtaa suoritettavan ohjelman

\textbf{sys\_time}\\
Numero: 17\\
Kutsuminen: \texttt{time\_t time(time\_t* timer)}\\
Implementaatio: \texttt{razos/kernel/src/syscall/sys\_time.c}\\
Kuvaus: palauttaa ajan sekunteina UNIXin Epochista, 1.1.1970 UTC00:00:00
