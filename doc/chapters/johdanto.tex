\chapter{Johdanto}
\label{Johdanto}

Käyttöjärjestelmiä on useita, joista tunnetuimpia ovat Microsoft Windows, Applen Mac OS X sekä erinäiset Linux-versiot. Linux ja Mac OS X pohjautuvat teknisesti 1970-luvulla kehitettyyn UNIX-käyttöjärjestelmään, jota kehittivät Bell Labs -tutkimuslaitoksessa Dennis Richie ja Ken Thompson. Tässä työssä tarkastellaan käyttöjärjestelmän tehtäviä ja teknistä toteutusta UNIXin näkökulmasta tämän työn ohessa kehitetyn UNIX-tyylisen käyttöjärjestelmän, RazOSin, avulla.

\section{Käyttöjärjestelmän tehtävät}
Käyttöjärjestelmä on ohjelmisto, joka mahdollistaa muiden ohjelmien toiminnan hallinnoimalla tietokoneen resursseja, kuten muistia. Se antaa ohjelmien käyttöön yhtenäisen rajapinnan, API:n, joka on riippumaton alustasta. Rajapintaan kuuluu monia järjestelmäkutsuja, jotka toimivat erinäisten abstrahointien, kuten tiedostojärjestelmän ja virtuaalimuistin, kanssa.

\par

UNIX-maailmassa The Open Groupin ja IEEE Computer Societyn julkaisema POSIX-standardi määrittelee järjestelmäkutsuja ja laajentaa niiden avulla International Organization for Standardizationin ja International Electrotechnical Commissionin standardisoimaa C-kirjastoa. Standardisoidun C-kirjaston avulla ohjelmoijien on mahdollista kehittää ohjelmia, jotka toimivat kaikissa standardien mukaisissa käyttöjärjestelmissä.

\par

Käyttöjärjestelmä myös pitää huolta käyttäjän ja ohjelmien turvallisuudesta. Se esimerkiksi tarkistaa, onko käyttäjällä ja/tai ohjelmalla oikeus käyttää pyytämäänsä tiedostoa, ja suojelee muiden ohjelmien muistialueita virtuaalimuistinhallinan keinoin, etteivät ohjelmat käytä muistia, johon niillä ei ole lupaa koskea.

\par

Virheidenhallinta ja niistä selviäminen ovat myös käyttöjärjestelmän tehtäviä. UNIXeissa virheidenhallinta on tavallisesti hoidettu \textit{signaaleilla}, joita käyttöjärjestelmä antaa C-kirjastolle virheen sattuessa. Kirjasto hoitaa yleensä virheet itse, usein lopettamalla ohjelman suorituksen. Käyttäjä voi myös itse määrittää signaaleille käsittelyfunktion ISO-standardiin kuuluvalla \texttt{signal()} -funktiolla.