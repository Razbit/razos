\chapter{Yhteenveto}

RazOS:in käytettävyydessä on vielä parannettavaa. Suurin puute on kunnollisen päätteen eli terminaaliemulaattorin puuttuminen. Tämä tekee näyttötulosteen tarkasta hallinnasta käyttäjätilassa vaikeaa. Pääteohjelma on kuitenkin jo olemassa, ja se tullaan käyttöjärjestelmän kehitystyön jatkuessa pian lisäämään osaksi järjestelmää.

\par

Unix tunnetaan ehkä parhaiten komentorivityökaluistaan, kuten komennoista grep, awk ja cat. RazOS:issa ei toistaiseksi ole näitä Unixin tuntomerkkejä, mutta niiden lisääminen osaksi järjestelmää on jo nyt täysin mahdollista. Lisäksi mikä tahansa POSIX-yhteensopiva toteutus näistä ja muista ohjelmista voi suoraan toimia myös RazOS:issa.

\par

Lähdekoodi kokonaisuudessaan löytyy GitHubista, osoiteesta \url{https://github.com/Razbit/razos}. Sieltä löytyy myös ohjeet RazOSin ``asentamiseen'' ja käyttämiseen, tiedostosta README.md. Oikealla tietokoneella käyttämistä emme suosittele.

\section{Ajankäyttö ja työnjako}

Projekti alkoi jo syksyllä 2014. Sen jälkeen on ollut kaksi täydellistä uudelleenkirjoitusta, ja nykyinen versio sai alkunsa syksyllä 2015. Talven aikana muistinhallinta meni kokonaan uusiksi. Aikaa on siis mennyt todella paljon. GitHubin statistiikat kertovat, että koodia on noin 18000 riviä.

\par

Iiro tuli projektiin mukaan vasta alkusyksystä 2015. Silloin oli kernelin toinen uudelleenkirjoitus alkamassa, ja tehtiin sellainen työnjako, että Iiro tekee käyttäjäpuolta ja Eetu kerneliä. Toki vähän meni ristiin välillä, mutta se oli ajatus ja siinä on melko hyvin pysytty.
