\chapter{Yhteenveto}

RazOS:in käytettävyydessä on vielä parannettavaa. Suurin puute on kunnollisen päätteen eli terminaaliemulaattorin puuttuminen. Tämä tekee näyttötulosteen tarkasta hallinnasta käyttäjätilassa vaikeaa. Pääteohjelma on kuitenkin jo olemassa, ja se tullaan käyttöjärjestelmän kehitystyön jatkuessa pian lisäämään osaksi järjestelmää.

\par

Unix tunnetaan ehkä parhaiten komentorivityökaluistaan, kuten komennoista grep, awk ja cat. RazOS:issa ei toistaiseksi ole näitä Unixin tuntomerkkejä, mutta niiden lisääminen osaksi järjestelmää on jo nyt täysin mahdollista. Lisäksi mikä tahansa POSIX-yhteensopiva toteutus näistä ja muista ohjelmista voi suoraan toimia myös RazOS:issa.

\par

Lähdekoodi kokonaisuudessaan löytyy GitHubista, osoiteesta \url{https://github.com/Razbit/razos}. Sieltä löytyy myös ohjeet RazOSin ``asentamiseen'' ja käyttämiseen, tiedostosta README.md. Oikealla tietokoneella käyttämistä emme suosittele. Jos tiedoston \texttt{floppy.img} laittaa johonkin emulaattoriin, kuten Qemuun, levykkeeksi, voi käyttöjärjestelmää kokeilla.

\section{Ajankäyttö ja työnjako}

Projekti alkoi jo syksyllä 2014. Sen jälkeen on ollut kaksi täydellistä uudelleenkirjoitusta, ja nykyinen versio sai alkunsa syksyllä 2015. Talven aikana muistinhallinta meni kokonaan uusiksi. Aikaa on siis mennyt todella paljon. GitHubin statistiikat kertovat, että koodia on noin 18000 riviä.

\par

Iiro tuli projektiin mukaan vasta alkusyksystä 2015. Silloin oli kernelin toinen uudelleenkirjoitus alkamassa, ja tehtiin sellainen työnjako, että Iiro tekee käyttäjäpuolta ja Eetu kerneliä. Toki vähän meni ristiin välillä, mutta se oli ajatus ja siinä on melko hyvin pysytty.

\section{Mitä opimme}

Jo ennen projektia meillä oli molemmilla suhteellisen kattava ohjelmointitausta; C oli jo tuttu kieli ja monia ohjelmointiprojekteja oli takana. Tämän kaltainen projekti toi ohjelmointiin aivan uudenlaisen näkökulman: ennen saattoi luottaa siihen, että kirjasto tarjosi kaikenlaisia valmiita funktioita ja tietotyyppejä, mutta käyttöjärjestelmää tehdessä ne on määriteltävä itse. Assymblyllä ohjelmointi oli vierasta, mutta se tuli hyvin tutuksi projektin aikana. Binäärimuotoon käännettyjen ohjelmien rakenteesta saimme valtavan määrän uutta tietoa, kuten myös laitteiston ja ohjelmiston toimimisesta yhdessä. Laitteiston läheinen ohjelmointi oli uutta siinä mittakaavassa, jossa sitä piti käyttöjärjestelmän alimpia osia tehdessä harjoittaa.

\par

Lopputulos, RazOSin toiminnalisuus kirjoitushetkellä, yllätti tekijänsäkin; se toimii yllättävän hyvin, vaikka monesti on meinannut usko loppua. Tekemistä vielä toki on, että pääsisimme tavoitteeseemme: käyttöjärjestelmään, jolla voisi jatkaa sen itsensä kehitystä.

\par

Mitä tekisimme toisin? Montakin asiaa. RazOS on pyritty pitämään mahdollisimman yksinkertaisena, eikä liialliseen optimointiin ole ryhdytty. Se näkyy esimerkiksi siinä, että muistinhallinta on melko hidas, ja heap-alueet fragmentoituvat helposti. Nämä asiat tosin ovat omalla tavallaan projektin tarkoituksen ulkopuolella; nykyiset algoritmit selittävät verrattain yksinkertaisesti käyttöjärjestelmän toimintaa. Ehkä suurin muutos, mikä tulisi tehdä, on virtuaalisen tiedostojärjestelmän toteutus siten, että se tukisi kansioita.