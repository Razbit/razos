%%%%%%%%%%%%%%%%%%%%%%%%%%%%%%%%%%%%%%%%%
% Heavily modified by Razbit from the following template:
%
% Masters/Doctoral Thesis
% LaTeX Template
% Version 2.2 (21/11/15)
%
% This template has been downloaded from:
% http://www.LaTeXTemplates.com
%
% Version 2.x major modifications by:
% Vel (vel@latextemplates.com)
%
% This template is based on a template by:
% Steve Gunn (http://users.ecs.soton.ac.uk/srg/softwaretools/document/templates/)
% Sunil Patel (http://www.sunilpatel.co.uk/thesis-template/)
%
% Template license:
% CC BY-NC-SA 3.0 (http://creativecommons.org/licenses/by-nc-sa/3.0/)
%
%%%%%%%%%%%%%%%%%%%%%%%%%%%%%%%%%%%%%%%%%

%----------------------------------------------------------------------------------------
%	PACKAGES AND OTHER DOCUMENT CONFIGURATIONS
%----------------------------------------------------------------------------------------

\documentclass[
11pt, % The default document font size, options: 10pt, 11pt, 12pt
%oneside, % Two side (alternating margins) for binding by default, uncomment to switch to one side
english,
singlespacing, % Single line spacing, alternatives: onehalfspacing or doublespacing
%draft, % Uncomment to enable draft mode (no pictures, no links, overfull hboxes indicated)
%nolistspacing, % Uncomment this to set spacing in lists to single
%liststotoc, % Uncomment to add the list of figures/tables/etc to the table of contents
%toctotoc, % Uncomment to add the main table of contents to the table of contents
parskip, % Uncomment to add space between paragraphs
nohyperref, % Uncomment to not load the hyperref package
headsepline, % Uncomment to get a line under the header
]{class} % The class file specifying the document structure

\usepackage[utf8]{inputenc} % Required for inputting international characters
\usepackage[T1]{fontenc} % Output font encoding for international characters

\usepackage{palatino} % Use the Palatino font by default

\usepackage{pageslts}
\usepackage[titletoc]{appendix}

\usepackage{minted}

\usepackage[backend=bibtex,style=authortitle,natbib=true]{biblatex}
\addbibresource{bibliography.bib} % The filename of the bibliography
\usepackage[autostyle=true]{csquotes} % Generate language-dependent quotes in the bibliography

%----------------------------------------------------------------------------------------
%	MARGIN SETTINGS
%----------------------------------------------------------------------------------------

\geometry{
	paper=a4paper, % Change to letterpaper for US letter
	inner=2.5cm, % Inner margin
	outer=3.8cm, % Outer margin
	bindingoffset=2cm, % Binding offset
	top=1.5cm, % Top margin
	bottom=1.5cm, % Bottom margin
	%showframe,% show how the type block is set on the page
}

%----------------------------------------------------------------------------------------
%	THESIS INFORMATION
%----------------------------------------------------------------------------------------

\thesistitle{UNIX-tyylinen käyttöjärjestelmäydin sekä ISO/IEC 9899:1999 ja POSIX.1-2008 -standardien mukainen C-kirjasto: RazOS} % \ttitle

\author{Pesonen, Eetu \& Rastas, Iiro} % \authorname
\webpage{http://github.com/razbit/razos} % \webpage
\keywords{Operating System, UNIX, POSIX, Programming} % \keywordnames
\newcommand\keywordnamesfi{Käyttöjärjestelmä, UNIX, POSIX, Ohjelmointi}
\university{Kerttulin Lukio} %  \univname
\department{ICT -linja} %  \deptname
\newcommand\nappendices{1} % Number of appendices. \nappendices

\abbrevnaming{Käytetyt merkinnät ja lyhenteet}

\begin{document}

\frontmatter % Use roman page numbering style (i, ii, iii, iv...) for the pre-content pages

\pagestyle{plain} % Default to the plain heading style until the thesis style is called for the body content

% For setting the chapter titles
\makeatletter
\def\@makechapterhead#1{
  \vspace*{10pt}
  {
    \reset@font\huge\bfseries
    \huge \thechapter{} \quad #1
    \par\nobreak
    \vskip 10pt
  }
}

\def\@makeschapterhead#1{%
  \vspace*{10pt}
  {
    \reset@font\huge\bfseries
    \huge #1
    \par\nobreak
    \vskip 10pt
  }
}


%----------------------------------------------------------------------------------------
%	TITLE PAGE
%----------------------------------------------------------------------------------------

\begin{titlepage}
\begin{center}

\textsc{\LARGE \univname}\\[1.5cm] % University name
\textsc{\Large ICT-linjan päättötyö}\\[0.5cm] % Thesis type

\doublespacing
{\huge \bf{\ttitle}}\\[0.4cm] % Thesis title
\singlespacing

Eetu Pesonen ja Iiro Rastas \\[0.4cm]

\deptname\\[2cm] % Research group name and department name

{\large \today}\\[4cm] % Date

\vfill
\end{center}
\end{titlepage}


%----------------------------------------------------------------------------------------
%	ABSTRACT PAGE
%----------------------------------------------------------------------------------------

\begin{abstract}
\addchaptertocentry{Tiivistelmä} % Add the abstract to the table of contents

\noindent
\MakeUppercase{\bf{\ttitle}}\\[0.4cm]
\authorname \\
\univname \\
\deptname \\
\today \\
Sivumäärä: \pageref{LastPage} (\lastpageref{LastPages})\\
Liitteitä: \nappendices \\
Asiasanat: \keywordnamesfi \\
\HRule

Käyttöjärjestelmät ovat monimutkaisia kokonaisuuksia. Työn tarkoitus on selventää käyttöjärjestelmän toimintaa ja toteutusta. Työssä kuvataan UNIX-tyylisen käyttöjärjestelmän teknistä toteutusta erityisesti prosessien, niiden hallinnan ja ajanjaon, muistinhallinnan sekä käyttäjän käytettävissä olevien rajapintojen kannalta. Teknisen toteutuksen kuvausta varten laadittiin UNIX-tyylinen käyttöjärjestelmä, RazOS. Toteutuksen kuvaukseen käytetään apuna lähdekoodia sekä teknisiä yksityiskohtia  selventäviä kaavioita.

\par

RazOS ohjelmoitiin C ja Assembly -ohjelmointikielillä, jotka ovat perinteisiä ohjelmointikieliä käyttöjärjestelmän ohjelmointiin. Alunperin C kehitettiin UNIX-käyttöjärjestelmän kehittämistä varten, joten se on erityisen sopiva ohjelmointikieli työn tarkoituksen kannalta. Alustaksi valittiin laajalti käytössä oleva Intelin x86 -arkkitehtuuri, koska yleisyytensä takia sille on helpompi laatia käyttöjärjestelmä kuin esimerkiksi ARM -arkkitehtuureille. Toinen syy x86:n valintaan oli sen ikä; sitä on käytetty lähes UNIXin alkuajoista asti, aina vuodesta 1978.

\par

Ohjelmointityön tuloksena saatiin aikaiseksi käyttöjärjestelmä, joka tukee osittain POSIX ja C99 -standardien mukaista C-kirjastoa. Sille tehtiin muutama ohjelma sekä komentokehote, rash. RazOSin kehitystyö jatkuu edelleen, ja tavoitteena on täysin standardien mukainen, UNIX-tyylinen käyttöjärjestelmä.

\end{abstract}

\clearpage

\begin{abstract}
\addchaptertocentry{Abstract} % Add the abstract to the table of contents

\noindent
\MakeUppercase{\bf{UNIX-like operating system kernel and ISO/IEC 9899:1999 and POSIX.1-2008 compliant C-library: RazOS}}\\[0.4cm]
\authorname \\
\univname \\
\deptname \\
\today \\
Number of pages: \pageref{LastPage} (\lastpageref{LastPages})\\
Appendices: \nappendices \\
Keywords: \keywordnames \\
\HRule

Ja sama englanniksi

\end{abstract}


%----------------------------------------------------------------------------------------
%	LIST OF CONTENTS/FIGURES/TABLES PAGES
%----------------------------------------------------------------------------------------
\renewcommand{\contentsname}{Sisällysluettelo}
\tableofcontents % Prints the main table of contents


%----------------------------------------------------------------------------------------
%	ABBREVIATIONS
%----------------------------------------------------------------------------------------

\begin{abbreviations}{ll} % Include a list of abbreviations (a table of two columns)

\textbf{API} & \textbf{A}pplication \textbf{P}rogramming \textbf{I}nterface, ohjelmointirajapinta\\

\end{abbreviations}


%----------------------------------------------------------------------------------------
%	THESIS CONTENT - CHAPTERS
%----------------------------------------------------------------------------------------

\mainmatter % Begin numeric (1,2,3...) page numbering

\pagestyle{thesis} % Return the page headers back to the "thesis" style

% Include the chapters of the thesis as separate files from the Chapters folder
% Uncomment the lines as you write the chapters

\chapter{Johdanto}
\label{Johdanto}

Käyttöjärjestelmiä on useita, joista tunnetuimpia ovat Microsoft Windows, Applen Mac OS X sekä erinäiset Linux-versiot. Linux ja Mac OS X pohjautuvat teknisesti 1970-luvulla kehitettyyn UNIX-käyttöjärjestelmään, jota kehittivät Bell Labs -tutkimuslaitoksessa Dennis Richie ja Ken Thompson. Tässä työssä tarkastellaan käyttöjärjestelmän tehtäviä ja teknistä toteutusta UNIXin näkökulmasta tämän työn ohessa kehitetyn UNIX-tyylisen käyttöjärjestelmän, RazOSin, avulla.

\section{Käyttöjärjestelmän tehtävät}
Käyttöjärjestelmä on ohjelmisto, joka mahdollistaa muiden ohjelmien toiminnan hallinnoimalla tietokoneen resursseja, kuten muistia. Se antaa ohjelmien käyttöön yhtenäisen rajapinnan, API:n, joka on riippumaton alustasta. Rajapintaan kuuluu monia järjestelmäkutsuja, jotka toimivat erinäisten abstrahointien, kuten tiedostojärjestelmän ja virtuaalimuistin, kanssa.

\par

UNIX-maailmassa The Open Groupin ja IEEE Computer Societyn julkaisema POSIX-standardi määrittelee järjestelmäkutsuja ja laajentaa niiden avulla International Organization for Standardizationin ja International Electrotechnical Commissionin standardisoimaa C-kirjastoa. Standardisoidun C-kirjaston avulla ohjelmoijien on mahdollista kehittää ohjelmia, jotka toimivat kaikissa standardien mukaisissa käyttöjärjestelmissä.

\par

Käyttöjärjestelmä myös pitää huolta käyttäjän ja ohjelmien turvallisuudesta. Se esimerkiksi tarkistaa, onko käyttäjällä ja/tai ohjelmalla oikeus käyttää pyytämäänsä tiedostoa, ja suojelee muiden ohjelmien muistialueita virtuaalimuistinhallinan keinoin, etteivät ohjelmat käytä muistia, johon niillä ei ole lupaa koskea.

\par

Virheidenhallinta ja niistä selviäminen ovat myös käyttöjärjestelmän tehtäviä. UNIXeissa virheidenhallinta on tavallisesti hoidettu \textit{signaaleilla}, joita käyttöjärjestelmä antaa C-kirjastolle virheen sattuessa. Kirjasto hoitaa yleensä virheet itse, usein lopettamalla ohjelman suorituksen. Käyttäjä voi myös itse määrittää signaaleille käsittelyfunktion ISO-standardiin kuuluvalla \texttt{signal()} -funktiolla.


%----------------------------------------------------------------------------------------
%	THESIS CONTENT - APPENDICES
%----------------------------------------------------------------------------------------

\begin{appendices}
\renewcommand\appendixname{Liite}
\appendix % Cue to tell LaTeX that the following "chapters" are Appendices

% Include the appendices of the thesis as separate files from the Appendices folder
% Uncomment the lines as you write the Appendices

\chapter{Liitteen otsikko}

\label{AppendixA} % For \ref{AppendixA}

Liite tähän

\end{appendices}
%----------------------------------------------------------------------------------------
%	BIBLIOGRAPHY
%----------------------------------------------------------------------------------------

\printbibliography[title={Kirjallisuus}, heading=bibintoc]

%----------------------------------------------------------------------------------------

\end{document}
